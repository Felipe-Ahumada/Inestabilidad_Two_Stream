\documentclass[11pt]{article}
\usepackage[utf8]{inputenc}
\usepackage[spanish]{babel}
\usepackage{graphicx}
\usepackage{array}
\usepackage{hyperref}
\usepackage{enumerate}
\usepackage{enumitem}
\usepackage[a4paper]{geometry}
\geometry{top=2.5cm, bottom=2.0cm,left=2.25cm, right=2.25cm}
\usepackage{amsmath,amsfonts,amssymb,graphicx,fancyhdr,vmargin,mathtools,nccmath,aligned-overset,cancel,nccmath}
\usepackage[most]{tcolorbox}

\makeatletter
\renewcommand*\env@matrix[1][*\c@MaxMatrixCols c]{ 
  \hskip -\arraycolsep
  \let\@ifnextchar\new@ifnextchar
  \array{#1}}
\makeatother
\tcbset{colback=yellow!10!white, colframe=red!50!black, 
        highlight math style= {enhanced, %<-- needed for the ’remember’ options
            colframe=blue,colback=red!10!white,boxsep=0pt}
            }

%\setlength{\lheadwidth}{60mm}
\setlength{\textwidth}{163mm}
\setlength{\topmargin}{10mm}
\setlength{\evensidemargin}{-1.5cm}
\setlength{\oddsidemargin}{2cm} %espacio de la izquierda
\setlength{\headheight}{40pt} %espacio de arriba

\pagestyle{fancy}
\fancyhf{}
\lhead{\begin{picture}(0,0)
\put(0,0){\includegraphics[width=30mm]{Epico}}\end{picture}}
\chead{Física compu ll}
\rhead{03-10-2021}
\cfoot{\thepage}
\renewcommand{\headrulewidth}{2pt}


\newcommand{\eat}[1]{}
\newcommand{\HRule}{\rule{\linewidth}{0.5mm}}
%\newcommand{\HRule}[]{\rule{\lheadwidth}{0.5}}




\begin{document}
\begin{titlepage}
\begin{center}

\includegraphics[width=0.9\textwidth]{holi}~\\[1cm]

\HRule \\[0.4cm]
 \LARGE 
  
  \textbf{Proyecto N°1:}\\[0.4cm]
   \textbf{\textit{''Particle in cell''}}\\[0.4cm]
    \emph{Física computacional II\\[0.4cm]}
\HRule \\[1.0cm]

{ \large

    Felipe Ahumada Silva [2020430411]\\[0.1cm]
}

{ \large
    Profesor: Roberto Navarro. \\
    Ayudante: Lorena Sepulveda.
}

\vfill
\textsc{\large Facultad de Cs. Físicas y Matemáticas,\\Universidad de Concepción} \\[0.4cm]
{\large 03 de octubre de 2021}
\end{center}
\end{titlepage}
\newpage

\section{Particle in cell}

Tienes una cierta caja en una dimensión, en una dimension uno tiene un espacio en una direccion x, y uno dice que empieza desde 0 hasta un cierto L (tamaño caja), uno dice que las particulas solo se pueden mover en esta caja y en esa caja uno tiene particulas dispersas por toda la caja, en el método particle in cell uno dice que va a dividir un espacio en varias celdas y a cada celda se le conoce cierto campo electrico (E1, E2,...,En) y uno solo conoce el campo electrico en la mitad de la celda, en esta celda uno va contando cuantas partículas hay en cada celda y por lo tanto en cada celda uno conoce cierta densidad (n0,n1,...), por ejemplo uno dice, 'en una caja hay 3 partículas y la densidad en esa caja sería $n_0 = \frac{3}{\Delta x}$', al final uno tiene un $n_i$ que va a representar la densidad en cierta posición $x_i$ ($n_i = n(x_i)$) el $x_i$ sería las mitades de las celdas.

\section{Campo eléctrico}

El campo eléctrico se calcula con la ecuación de Gauss:

\begin{equation*}
    \nabla \cdot \overrightarrow{E} = n-n_0
\end{equation*}

La divergencia del campo eléctrico es igual a la densidad, donde n es la densidad de electrones ($n(x_i)$) y el $n_0$ es un background, es la cantidad de protones y uno dice que $n_0=1$, uno dice eso porque los protones son masivos en comparación que los electrones (son como moscas moviendose alrededor de un auto, por lo tanto, simplemente los protones no se mueven.\\

    En una dimensión la divergencia del campo eléctrico viene dada por:
    
\begin{equation*}
    \nabla \cdot \overrightarrow{E} = \frac{\partial E_x}{\partial x}
\end{equation*}
asi nuestra ecuación a resolver es:

\begin{equation*}
    \nabla \cdot \overrightarrow{E} = n - 1
\end{equation*}

Notemos que n depende del espacio ya que nos preguntamos cual es la densidad en cada celda. Así, podemos realizar una aproximación:

\begin{equation*}
    \frac{E_x(x_{i+1})-E(x_{i-1})}{2\Delta x} = n(x_i)-1 %Esto es una derivada centrada
\end{equation*}

Acá podemos despejar el primer término

\begin{equation*}
    E_x(x_{i+1}) = E(x_{i-1}) - 2\Delta x(n(x_i)-1)
\end{equation*}

Esto nos dice que si uno conoce $E_0$ puedes conocer $E_2$ por ejemplo, esto es el método conocido como el salto de la rana

\section{Las partículas}

Podemos escribir nuestras partículas como:

\begin{align*}
    \frac{dx}{dt} &= v \\
    \frac{dv}{dt} &= E
\end{align*}

Usando el método del salto de la rana:

\begin{align*}
    x(t_{i+1}) &= x(t_i) + \Delta t v(t_{i+1/2})\\
    v(t_{i+1/2}) &= v(t_{i-1/2}) + \Delta t E(t_i)
\end{align*}

Notemos que el campo depende del tiempo y del espacio, entonces, $E_x$ significa el campo eléctrico en cada una de las celdas al mismo tiempo.

\section{Detalles}

Eso de que pasa si el campo empuja una partícula fuera del espacio

\end{document}